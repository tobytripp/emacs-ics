\chapter{Other Monads}

The @Monad@ type\cite[p.~402]{thompson99}:

\begin{code}
class Monad m where
  (>>=)  :: m a -> (a -> m b) -> m b
  return :: a -> m a
  (>>)   :: m a -> m b -> m b
  fail   :: String -> m a
\end{code}

@Monad@ comes with some default definitions:

\begin{code}
  m >> k = m >>= \_ -> k
  fail s = error s
\end{code}

\begin{quote}
From this definition it can be seen that @>>@ acts like @>>=@, except that the
value returned by the first argument is discarded rather than being passed to
the second argument. \cite[p.~403]{thompson99}
\end{quote}

\section{The Identity Monad \cite[p.~404]{thompson99}}

The identity monad takes a type to itself with definitions:

\begin{code}
  m >>= f = f m
  return  = id
\end{code}

\section{Definition of Maybe}

\begin{code}
instance Monad Maybe where
  (Just x) >>= k  = k x
  Nothing >>= k   = Nothing
  return          = Just
  fail s          = Nothing
\end{code}

\section{The List \textit{Monad}}

\begin{code}
instance Monad [] where
  xs >>= f = concat (map f xs)
  return x = [x]
  fail s = []
\end{code}

Lists are, in fact, themselves instances of @Monad@.

\begin{code}
fmap  :: Functor f => (a -> b) -> f a -> f b
(>>=) :: Monad   m => m a -> (a -> m b) -> m b
map   :: (a -> b) -> [a] -> [b]
\end{code}
