\chapter{The Maybe Monad}

\section{As Described by Computerphile}

\autocite{hutton:computerphile}

\begin{code}
-- Type and 2 type constuctors: Val and Div
data Expr = Val Int | Div Expr Expr

-- Val 1
-- Div (Val 6) (Val 2)

unsafe_eval :: Expr -> Int
unsafe_eval (Val n) = n
unsafe_eval (Div x y) = div (eval x) (eval y)

-- eval (Div (Val 6) (Val 2))
\end{code}

What if @Div@ is passed zero?  The program will crash. So, error-checking is
necessary.

\begin{code}
safediv :: Int -> Int -> Maybe Int
safediv n m = if m == 0 then Nothing else Just (div n m)

eval :: Expr -> Maybe Int
eval (Val n)   = Just n
eval (Div x y) = case eval' x of
                   Nothing -> Nothing
                   Just n  -> case eval y of
                             Nothing -> Nothing
                             Just m  -> safediv n m
\end{code}

Abstracting the pattern of @case@ checking @Maybe@ values can be represented
as:

\begin{table}[h]
\begin{tabular}[b]{p{0.45\linewidth} | p{0.45\linewidth}}
\hline
Without `do' notation & With `do' notation \\ \hline
\begin{code}
eval' :: Expr -> Maybe Int
eval' (Val n)   = return n
eval' (Div x y) =
  eval' x >>= (\n ->
    eval' y >>= (\m ->
      safediv n m))
\end{code} &
\begin{code}
eval'' :: Expr -> Maybe Int
eval'' (Val n)   = return n
eval'' (Div x y) = do
  n <- eval'' x
  m <- eval'' y
  safediv n m
\end{code}
\end{tabular}
\end{table}
