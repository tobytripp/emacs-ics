% -*- coding: utf-8 -*-
\chapter{``Brian Beckman: Don't Fear the Monad''}

Available on \href{https://youtu.be/ZhuHCtR3xq8}{Youtube} \cite{beckman2012}.

Dr. Beckman, astrophysicist and senior software engineer, begins with a basic
introduction to functional programming as a concept.  Most notably, he focuses
on the concept of functions as being \textbf{replaceable} by table-lookups.

\section{Outline (\href{https://youtu.be/ZhuHCtR3xq8?t=7m50s}{7:50})}

\begin{enumerate}
\item Functions
\item Monoids
\item Functions
\item Monads
\end{enumerate}

\section{Notation (\href{https://youtu.be/ZhuHCtR3xq8?t=7m50s}{8:25})}

\begin{description}
\item [From ``imperative'' to functional notation] :
  \begin{itemize}
  \item $int\;x= x \in int$
  \item \codeline{x :: int}
  \end{itemize}
  \begin{itemize}
  \item $int\;f(int\;x)$
  \item \codeline{f :: int -> int}
  \end{itemize}
\item [Given type variable $a$ :] $\forall a$
  \begin{itemize}
  \item \codeline[language=Java]{A x}
  \item \codeline[language=Haskell]{x :: a}
  \end{itemize}
  \begin{itemize}
  \item \codeline[language=Java]{static A f<A>(A x)}
  \item \codeline[language=Haskell]{f :: a -> a}
  \end{itemize}
\end{description}

\subsection{Composition}

Given:
\begin{lstlisting}[frame=trBL, frameround=fttt, framerule=1pt, label=def]
  x :: a
  f :: a -> a
  g :: a -> a
\end{lstlisting}
in imperative style function composition might appear as: \codeline{f(g(a))}
or in reverse: \codeline{g(f(a))}.

In functional style, function application appears as: \codeline{g a} and
composition can be shown as: \codeline{f(g a)}.  Parenthesis are necessary due to
partial application being left associative.  For example, \codeline{f h g} is
applied as though \codeline{(f h) g}.

It is also possible to use a composition operator, $\circ$, to imply composition: \codeline{(f . g) a}.  So, given the above\ref{def}, we can deduce:
\begin{lstlisting}
  h = (f . g) a = f . g
  h :: a -> a
\end{lstlisting}
This does confuse the concepts of @a@ as argument and @a@ as type, but the
point remains clear, I think.


\section{Monoids (\href{https://youtu.be/ZhuHCtR3xq8?t=7m50s}{20:40})}

\begin{minipage}{\linewidth}
  \begin{quotation}
    In abstract algebra, a branch of mathematics, a monoid is an algebraic
    structure with a single associative binary operation and an identity
    element.

    Monoids are studied in semigroup theory, because they are semigroups with
    identity.

    \footnote{\href{https://en.wikipedia.org/wiki/Monoid}{Wikipedia}}:
  \end{quotation}
\end{minipage}


A @Monoid@ is a @Set@ with:
\begin{enumerate}
\item an associative binary operator (generally composition)
\item an identity value
\end{enumerate}

The operator need not be commutative.

In a programming context, a @Monoid@ guarantees type-consistency over function composition.

\section{Monads (\href{https://youtu.be/ZhuHCtR3xq8?t=7m50s}{30:39})}

Given:
\begin{lstlisting}[frame=trBL, framerule=1pt, label=defM]
  x :: a
  f :: a -> M a
  g :: a -> M a
  g :: a -> M a
\end{lstlisting}
@M@ is described as a ``Type Constructor.''

Again, Dr. Beckman is using @a@ to represent both a value of type @a@ as well
as the type itself @a@.  Here he introduced the @Monad@ ``bind'' operator:
@>>=@, which he likes to call ``shove'':
\begin{lstlisting}
  f :: a -> M a
  g :: a -> M a
  -- the right hand side is g, but written with
  -- a lambda to preserve symmetry
  \a -> (f a) >>= \a -> (g a)
\end{lstlisting}

The reason to preserve symmetry in the above expression is that the desired expression is ``bracketed'' as:
\begin{math}
  \lambda a \rightarrow [(f a) >>= \lambda a \rightarrow (g a)]
\end{math}
because the bind operator has type:
\begin{lstlisting}
  (>>=) :: Monad m => m a -> (a -> m b) -> m b
\end{lstlisting}
That is, @>>=@ accepts a @Monad@ (@M a@) and returns a function from
@a -> M a@.

The functions @f@, and @g@ live in a @Monoid@.  @M a@ (the \emph{data}) lives
in a @Monad@.

@(>>=)@ is the analog of function composition and, therefore, obeys the rules of a @Monoid@.  Including associativity and identity.

In a @Monad@, identity is--in Haskell--written as:
\begin{lstlisting}
  return :: Monad m => a -> m a
\end{lstlisting}

Extended to non-uniform types:
\begin{lstlisting}
  g :: a -> Mb
  f :: b -> Mc
  \a -> (g a) >>= \b -> (f b) :: a -> Mc
  g >>= \b -> (f b)
\end{lstlisting}
